\documentclass{article}

\usepackage[spanish]{babel}
\usepackage[utf8]{inputenc}
\usepackage{blindtext}
\usepackage{hyperref}
\usepackage{graphicx}
\usepackage{caption}
\usepackage[margin = 0.75in]{geometry}
\usepackage{parskip}
\usepackage{enumitem}
\usepackage{booktabs}
\usepackage{algpseudocode}
\usepackage{algorithmicx}

% \usepackage{minted}

\usepackage{sectsty}
\allsectionsfont{\normalfont\sffamily}
\newenvironment{allintypewriter}{\ttfamily}{\par}

\author{Daniel E. Cordovés Borroto C411 \\ \\ Email: \href{mailto:daniel.cordovesb@estudiantes.matcom.uh.cu}{daniel.cordovesb@estudiantes.matcom.uh.cu} \\ \\
Github: \href{https://github.com/dcordb/kojos-kitchen/}{https://github.com/dcordb/kojos-kitchen/} }
\title{Kojo's Kitchen}

\begin{document}
	\maketitle
	\newpage
    
    \section*{Problema a resolver}
        La cocina de Kojo es uno de los puestos de comida rápida en un centro
        comercial. El centro comercial está abierto entre las 10:00 am y las 9:00 pm cada
        dı́a.
        
        En este lugar se sirven dos tipos de productos: sándwiches y sushi. Para los
        objetivos de este proyecto se asumirá que existen solo dos tipos de consumidores:
        unos consumen solo sándwiches y los otros consumen solo productos de la gama
        del sushi.
        
        En Kojo hay dos períodos de hora pico durante un día de trabajo;
        uno entre las 11:30 am y la 1:30 pm, y el otro entre las 5:00 pm y las 7:00
        pm. El intervalo de tiempo entre el arribo de un consumidor y el de otro no es
        homogéneo pero, por conveniencia, se asumirá que es homogéneo. El intervalo
        de tiempo de los segmentos homogéneos, distribuye de forma exponencial.

        Actualmente dos empleados trabajan todo el dı́a preparando sándwiches y
        sushi para los consumidores. El tiempo de preparación depende del producto en
        cuestión. Estos distribuyen de forma uniforme, en un rango de 3 a 5 minutos
        para la preparación de sándwiches y entre 5 y 8 minutos para la preparación de
        sushi.

        El administrador de Kojo está muy feliz con el negocio, pero ha estado recibiendo
        quejas de los consumidores por la demora de sus peticiones. Él está interesado
        en explorar algunas opciones de distribución del personal para reducir
        el número de quejas. Su interés está centrado en comparar la situación actual con
        una opción alternativa donde se emplea un tercer empleado durante los perı́odos
        más ocupados. La medida del desempeño de estas opciones estará dada por el
        porciento de consumidores que espera más de 5 minutos por un servicio durante
        el curso de un dı́a de trabajo.

        Se desea obtener el porciento de consumidores que esperan más de 5 minutos
        cuando solo dos empleados están trabajando y este mismo dato agregando un
        empleado en las horas pico.

    \section*{Solución}
        Para darle solución a este problema se realizan dos simulaciones: una con solo dos empleados; y otra con dos empleados y un tercero
        en los horarios pico, luego se comparan los porcientos de clientes que esperan más de 5 minutos en cada simulación para sacar
        conclusiones sobre la eficiencia en Kojo's Kitchen.

        Se tienen las siguientes variables:

        De tiempo:

        \begin{itemize}[label=\textbullet]
            \item \texttt{t} (tiempo actual en el sistema)
        \end{itemize}

        Estado del sistema:

        \begin{itemize}[label=\textbullet]
            \item \texttt{n}: número de personas en el sistema
            \item \texttt{who\_first}: número de cliente que está siendo atendido por el empleado 1 (0 indica que no hay cliente siendo atendido)
            \item \texttt{who\_second}: número de cliente está siendo atendido por el empleado 2 (0 indica que no hay cliente siendo atendido)
            \item \texttt{who\_third}: número de cliente está siendo atendido por el empleado 3 (0 indica que no hay cliente siendo atendido)
            \item \texttt{t\_out\_1}: tiempo después del cual el cliente \texttt{who\_first} es atendido por el empleado 1 (infinito si no hay cliente siendo atendido)
            \item \texttt{t\_out\_2}: tiempo después del cual el cliente \texttt{who\_second} es atendido por el empleado 2 (infinito si no hay cliente siendo atendido)
            \item \texttt{t\_out\_3}: tiempo después del cual el cliente \texttt{who\_third} es atendido por el empleado 3 (infinito si no hay cliente siendo atendido)
        \end{itemize}

        Contadoras:

        \begin{itemize}[label=\textbullet]
            \item \texttt{arrivals}: cantidad de llegadas
            \item \texttt{first\_empl\_req}: cantidad de pedidos completados por empleado 1
            \item \texttt{second\_empl\_req}: cantidad de pedidos completados por empleado 2
            \item \texttt{third\_empl\_req}: cantidad de pedidos completados por empleado 3
        \end{itemize}

        De salida:

        \begin{itemize}[label=\textbullet]
            \item \texttt{t\_in}: tiempo de llegada de cada cliente
            \item \texttt{t\_out}: tiempo de salida de cada cliente
            \item \texttt{food\_wants}: tipo de comida que quiere cada cliente
        \end{itemize}

        Los eventos serían:

        \begin{itemize}[label=\textbullet]
            \item llegada de un nuevo cliente
            \item primer empleado terminó pedido de algún cliente
            \item segundo empleado terminó pedido de algún cliente
            \item tercer empleado terminó pedido de algún cliente (nota: este empleado solo trabaja en horarios pico)
        \end{itemize}

        En la solución la unidad de tiempo es minutos. Por consiguiente se reescriben los intervalos de tiempo de esta forma:

        \begin{itemize}[label=\textbullet]
            \item tiempo en que abre la cocina: \texttt{t} = 0 (10:00 am)
            \item tiempo de inicio de primer intervalo de hora pico: \texttt{t} = 90 (11:30 am)
            \item tiempo de finalización de primer intervalo de hora pico: \texttt{t} = 90 + 120 (1:30 pm)
            \item tiempo de inicio de segundo intervalo de hora pico: \texttt{t} = 7 * 60 (5:00 pm)
            \item tiempo de finalización de segundo intervalo de hora pico: \texttt{t} = 7 * 60 + 120 (7:00 pm)
            \item tiempo de cierre de la cocina: \texttt{t} = 11 * 60 (9:00 pm)
        \end{itemize}

        Las llegadas de un nuevo cliente se realizan de la siguiente forma: se genera una variable con distribución exponencial
        que denota que a partir del tiempo actual (\texttt{t}) entra un nuevo cliente en G minutos (el tiempo de entrada sería \texttt{t + G} minutos),
        donde \texttt{G} es la variable generada.

        Como hay intervalos de tiempo de hora pico e intervalos de tiempo regulares se tienen dos lambdas distintos para generar
        estas variables:
            \begin{itemize}[label=\textbullet]
                \item \texttt{LAMBDA\_REGULAR} (lambda a usar en intervalos regulares)
                \item \texttt{LAMDA\_BUSY} (lambda a usar en intervalos de hora pico)
            \end{itemize}

        Por supuesto estos lambdas cumplen que \texttt{LAMBDA\_BUSY} $>$ \texttt{LAMBDA\_REGULAR}.

        Luego que llega un cliente se le asigna al primer empleado que lo pueda atender. Notar que el tercer empleado solo
        trabaja en la segunda simulación y en horarios picos. Cuando un empleado atiende a un cliente, el cliente sale del sistema y
        el empleado se le asigna un nuevo cliente, en particular, se le asigna el cliente $\max($\texttt{who\_first}$,$ \texttt{who\_second}$,$ \texttt{who\_third}$) + 1$.

        El tiempo que un cliente $i$ tuvo que esperar hasta recibir su comida es de \texttt{t\_out[i]} - \texttt{t\_in[i]}, queremos en cada simulación
        hallar el porciento de clientes cuyo tiempo de espera es mayor que 5 minutos.

    \section*{Consideraciones Obtenidas}
        Luego de correr varias simulaciones variando los parámetros \texttt{LAMBDA\_REGULAR} y \texttt{LAMBDA\_BUSY} se obtuvieron los siguientes resultados:

        \begin{center}
            \begin{tabular}{ccccc}
                \toprule
                Simulación   & \texttt{LAMBDA\_REGULAR} & \texttt{LAMBDA\_BUSY} & Two Employees & Three Employees \\ \midrule
                1            & 0.2                      & 0.3                   & 78.34394      & 68.71165        \\
                2            & 0.2                      & 0.3                   & 77.84810      & 54.30463          \\
                3            & 1                        & 0.7                   & 99.66159      & 66.25222          \\
                4            & 0.1                      & 0.05                  & 51.51515      & 58.13953          \\
                5            & 0.1                      & 0.05                  & 55.81395      & 44.00000          \\ \bottomrule
            \end{tabular}
        \end{center}

        Podemos notar que la cantidad de clientes que esperan más de 5 minutos es menor usando un tercer empleado. Sin embargo existen simulaciones
        en las que no es así (o están muy cercanos los porcientos), esto es entendible ya que la simulación con tres empleados pudo haber tenido más clientes
        entrando a la cocina en distintos momentos de tiempo. Claro esto es una situación pesimista, en general el modelo de tres empleados minimiza el
        tiempo de espera de los clientes.

        Para más detalles sobre el modelo, realizar simulaciones con otros parámetros lambda y ver el código que lo implementa dirigirse al repo de GitHub:

        \href{https://github.com/dcordb/kojos-kitchen/}{https://github.com/dcordb/kojos-kitchen/}

        \section*{Pseudocódigo}
        \begin{allintypewriter}
            \begin{algorithmic}[1]
                \Function{KojosKitchen}{use\_extra\_employee}
                    \While{True}
                        \State cur\_time = $\min$(next\_arrival, t\_out\_1, t\_out\_2, t\_out\_3)
                        % \[\]

                        \If{next\_arrival $\neq$ inf \textbf{and} next\_arrival = cur\_time}
                            \State t $\gets$ next\_arrival
                            \State gen food preference
                            \State t\_in $\gets$ t\_in + t \Comment {esto es concatenación}
                            \State arrivals $\gets$ arrivals + 1
                            % \[\]

                            \If {who\_first = 0}
                                \State n $\gets$ n + 1
                                \State who\_first $\gets$ arrivals
                                \State t\_out\_1 $\gets$ t + $T_i$ \Comment{$T_i$ es el tiempo que el empleado tarda}

                            % \[\]
                            \ElsIf{who\_second = 0}
                                \State n $\gets$ n + 1
                                \State who\_second $\gets$ arrivals
                                \State t\_out\_2 $\gets$ t + $T_i$ \Comment{$T_i$ es el tiempo que el empleado tarda}
                            
                            % \[\]
                            \ElsIf{use\_extra\_employee \textbf{and} in\_busy\_hours(t) \textbf{and} who\_third = 0}
                                \State n $\gets$ n + 1
                                \State who\_third $\gets$ arrivals
                                \State t\_out\_3 $\gets$ t + $T_i$ \Comment{$T_i$ es el tiempo que el empleado tarda}
                            
                            % \[\]
                            \Else
                                \Comment{Cliente debe esperar en la cola}
                                \State \textbf{pass}
                            \EndIf
                            % \[\]

                            \If{in\_busy\_hours(t)}
                                \State next\_arrival $\gets$ t + $G_{busy}$ \Comment{$G_{busy}$ es una exponencial generada con LAMBDA\_BUSY}
                            \Else
                                \State next\_arrival $\gets$ t + $G_{reg}$ \Comment{$G_{reg}$ es una exponencial generada con LAMBDA\_REGULAR}
                            \EndIf
                            
                            % \[\]

                            \If{next\_arrival $>$ T} \Comment{El nuevo cliente llega luego de cerrar la cocina}
                                \State next\_arrival $\gets$ inf
                            \EndIf

                        % \[\]
                        \ElsIf{t\_out\_1 $\neq$ inf \textbf{and} t\_out\_1 = cur\_time}
                            \State first\_empl\_req $\gets$ first\_empl\_req + 1
                            \State t $\gets$ t\_out\_1
                            \State t\_out[who\_first] = t

                            \State n $\gets$ n - 1
                            \State next\_guy $\gets$ $\max$(who\_first, who\_second, who\_third) + 1
                            \State who\_first $\gets$ 0 

                            \If{next\_guy $\leq$ arrivals}
                                \State who\_first $\gets$ next\_guy
                            \EndIf

                            \If{who\_first = 0}
                                \State t\_out\_1 $\gets$ inf

                            \Else
                                \State t\_out\_1 $\gets$ t + $T_i$ \Comment{$T_i$ es el tiempo que el empleado tarda}
                            \EndIf

                            % \[\]
                        \ElsIf{t\_out\_2 $\neq$ inf \textbf{and} t\_out\_2 = cur\_time}
                            \State second\_empl\_req $\gets$ second\_empl\_req + 1
                            \State t $\gets$ t\_out\_2
                            \State t\_out[who\_second] = t

                            \State n $\gets$ n - 1
                            \State next\_guy $\gets$ $\max$(who\_first, who\_second, who\_third) + 1
                            \State who\_second $\gets$ 0 

                            \If{next\_guy $\leq$ arrivals}
                                \State who\_second $\gets$ next\_guy
                            \EndIf

                            \If{who\_second = 0}
                                \State t\_out\_2 $\gets$ inf

                            \Else
                                \State t\_out\_2 $\gets$ t + $T_i$ \Comment{$T_i$ es el tiempo que el empleado tarda}
                            \EndIf

                            % \[\]
                        \ElsIf{use\_extra\_employee \textbf{and} t\_out\_3 $\neq$ inf \textbf{and} t\_out\_3 = cur\_time}
                            \State third\_empl\_req $\gets$ third\_empl\_req + 1
                            \State t $\gets$ t\_out\_3
                            \State t\_out[who\_third] = t

                            \State n $\gets$ n - 1
                            \State next\_guy $\gets$ $\max$(who\_first, who\_second, who\_third) + 1
                            \State who\_third $\gets$ 0 

                            \If{next\_guy $\leq$ arrivals}
                                \State who\_third $\gets$ next\_guy
                            \EndIf

                            \If{who\_third = 0}
                                \State t\_out\_3 $\gets$ inf

                            \Else
                                \State t\_out\_3 $\gets$ t + $T_i$ \Comment{$T_i$ es el tiempo que el empleado tarda}
                            \EndIf
                            % \[\]
                        \Else
                            \State \textbf{break}
                        \EndIf
                    \EndWhile

                    \State cnt $\gets$ 0
                    \ForAll{$1 \leq i \leq $ arrivals}
                        \If {t\_out[i] - t\_in[i] $>$ 5}
                            \State cnt $\gets$ cnt + 1
                        \EndIf
                    \EndFor

                    \State \textbf{return} cnt * 100 / arrivals
                \EndFunction
            \end{algorithmic}
        \end{allintypewriter}
\end{document}